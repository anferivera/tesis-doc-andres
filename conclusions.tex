\section{For the SDFDM model with scalars}
We have combined the singlet-doublet fermion dark matter (SDFDM) and
the singlet scalar dark matter (SSDM) models into a framework that generates radiative neutrino masses. 
The required lepton number violation only happens if the scalars are
real.   
We have then explored the novel features of the final model in flavor
physics, collider searches, and dark matter related experiments.  
In the case of SSDM, for example, the singlet-doublet fermion mixing
cannot be too small in order to be compatible with lepton flavor
violating (LFV) observables like $\operatorname{Br}(\mu\to e\gamma)$,
while in the case of fermion dark matter the LFV constraints are
automatically satisfied.
The presence of new decay channels for the next to lightest odd
particle opens the possibility of new signals at the LHC.
In particular, when the singlet scalar is the lightest
odd-particle and the singlet-like Majorana fermion is heavier than the
charged Dirac fermion, the production of the later yields dilepton plus missing transverse energy signals. For large enough
$e^\pm$ or $\mu^\pm$ branchings, these signals could exclude charged
Dirac fermion masses
of order $\SI{500}{GeV}$ in the Run I of the LHC. 
Finally, the effect of coannihilations with the scalar singlets was
studied in the case of doublet-like fermion dark matter.  In that
case, it is possible to obtain the observed dark matter relic density
with lower values of the LOP mass.


\section{For the GCE}
In this work we have entertained the possibility of finding model points in the SDFDM model that can explain the GCE while being in agreement with a multitude of different direct and indirect DM detection constrains. We found two viable regions: (i) DM particles present in the model with masses of $\sim 99$ GeV annihilating mainly into $W$ bosons with branching ratios greater than $\sim 70\%$, (ii) and a second region where the DM particle mass is in the range $\sim (173-190)$ GeV annihilating predominantly into the $t\bar{t}$ channel with branching ratios greater than $\sim 90\%$. Our analysis assumed that the DM is made entirely out of the lightest stable particle $\chi^0$ of the SDFDM model. Despite this being a very restrictive assumption, we have demonstrated that there exist models capable of accounting for the GeV excess in the GC that can be fully tested by the forthcoming XENON-1T and LZ experiments as well as by future \textit{Fermi}-LAT observations in dwarf galaxies. Interestingly, the most recent limits presented by LUX are able to probe a fraction of the good fitting models to the GCE found in this work. We also showed through realistic calculations of CTA performance when observing the GC that this instrument will not have the ability to confirm the SDFDM model if it is causing the GCE.