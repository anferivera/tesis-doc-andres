
\section{The interaction's Lagrangian}
\label{sec:lag-interaction}

In the SDFDM model the interaction's Lagrangian is given by
%
\begin{align}
\label{eq:lint1}
\mathcal{L}_{\text{Int}}
= & \dfrac{1}{2}\bar{N}i\slashed{\partial}N + \bar{R_d}i\slashed{D}R_d + \bar{\widetilde{R}_u}i\slashed{D}\widetilde{R}_u
- \dfrac{h}{\sqrt{2}}\left(-\lambda_d\psi_L^0N + \lambda_u{\psi_R^0}^{\dagger}N+\text{h.c}\right)\,.
\end{align}

If we defined, 
\begin{align}
\label{eq:devcov}
\slashed{D}=&\gamma^{\mu}\left(\partial_{\mu}+ig\widetilde{W}_{\mu}(x)+ig'yB_{\mu}(x)\right)\nonumber\\
=&\gamma^{\mu}\left(\partial_{\mu}
+ig\dfrac{1}{2}\begin{pmatrix}
W_{\mu}^3 & \sqrt{2}W_{\mu}^{\dagger} \\
\sqrt{2}W_{\mu} & -W_{\mu}^3
\end{pmatrix}
-ig'\dfrac{1}{2} B^{\mu}(x)\right)
\end{align}
%
as the covariant derivative, then we have to construct the four components Dirac spinors in therms of the two components Weyl spinors that we have in the SDFDM model. 
Therefore, in the following,  we will consider the Dirac spinors $\Psi_{L}^0=(\psi_L^0,0)^{T}=P_{L}\Psi_L^0$ and $\Psi_{L}^-=(\psi_L^-,0)^{T}=P_{L}\Psi_L^-$  of four components that we will construct later explicitly in terms of the Majorana spinors. 

Thus in the Eq.~\eqref{eq:lint1} we get explicitly that
\begin{align}
\label{eq:kinu}
\bar{R_d}i\slashed{D}R_d 
=&\bar{R_d}i\gamma^{\mu}\left(\partial_{\mu}
+ig\dfrac{1}{2}\begin{pmatrix}
W_{\mu}^3 & \sqrt{2}W_{\mu}^{\dagger} \\
\sqrt{2}W_{\mu} & -W_{\mu}^3
\end{pmatrix}
-ig'\dfrac{1}{2} B_{\mu}(x)\right)R_d\nonumber\\
=&\begin{pmatrix}
\bar{\Psi}_L^0 & \bar{\Psi}_L^-
\end{pmatrix} i\gamma^{\mu}\left(\partial_{\mu}
+ig\dfrac{1}{2}\begin{pmatrix}
W_{\mu}^3 & \sqrt{2}W_{\mu}^{\dagger} \\
\sqrt{2}W_{\mu} & -W_{\mu}^3
\end{pmatrix}
-ig'\dfrac{1}{2} B_{\mu}(x)\right)\begin{pmatrix}
\Psi_L^0 \\ \Psi_L^-
\end{pmatrix}\nonumber\\
=&\left(\bar{\Psi}_L^0i\slashed{\partial}\Psi_L^0+\bar{\Psi}_L^-i\slashed{\partial}\Psi_L^-\right)
-\dfrac{g}{2}\left[\bar{\Psi}_L^0\gamma^{\mu}W_{\mu}^3\Psi_L^0+\sqrt{2}\bar{\Psi}_L^-\gamma^{\mu}W_{\mu}\Psi_L^0
+\sqrt{2}\bar{\Psi}_L^0\gamma^{\mu}W_{\mu}^{\dagger}\Psi_L^--\bar{\Psi}_L^-\gamma^{\mu}W_{\mu}^3\Psi_L^-\right]
\nonumber\\
+&\dfrac{g'}{2}\left(\bar{\Psi}_L^0\gamma^{\mu}B_{\mu}\Psi_L^0+\bar{\Psi}_L^-\gamma^{\mu}B_{\mu}\Psi_L^-\right)
\nonumber\\
=&\left(\bar{\Psi}_L^0i\slashed{\partial}\Psi_L^0+\bar{\Psi}_L^-i\slashed{\partial}\Psi_L^-\right)
-\dfrac{g}{\sqrt{2}}\left[\bar{\Psi}_L^-\gamma^{\mu}W_{\mu}\Psi_L^0
+\text{h.c}\right] - \dfrac{g}{2}\left[\bar{\Psi}_L^0\gamma^{\mu}W_{\mu}^3\Psi_L^0-\bar{\Psi}_L^-\gamma^{\mu}W_{\mu}^3\Psi_L^-\right]
\nonumber\\
+&\dfrac{g'}{2}\left(\bar{\Psi}_L^0\gamma^{\mu}B_{\mu}\Psi_L^0+\bar{\Psi}_L^-\gamma^{\mu}B_{\mu}\Psi_L^-\right)\nonumber\\
=&\left(\bar{\Psi}_L^0i\slashed{\partial}\Psi_L^0+\bar{\Psi}_L^-i\slashed{\partial}\Psi_L^-\right)
-\dfrac{g}{\sqrt{2}}\left(\bar{\Psi}_L^-\gamma^{\mu}W_{\mu}\Psi_L^0
+\text{h.c}\right)\nonumber\\
-& \dfrac{1}{2}\left[\bar{\Psi}_L^0\gamma^{\mu}(gW_{\mu}^3-g'B_{\mu})\Psi_L^0-\bar{\Psi}_L^-\gamma^{\mu}(gW_{\mu}^3+g'B_{\mu})\Psi_L^-\right]
\nonumber\\
=&\left(\bar{\Psi}_L^0i\slashed{\partial}\Psi_L^0+\bar{\Psi}_L^-i\slashed{\partial}\Psi_L^-\right)
-\dfrac{g}{\sqrt{2}}\left(\bar{\Psi}_L^-\gamma^{\mu}W_{\mu}\Psi_L^0
+\text{h.c}\right)\nonumber\\
-&\dfrac{1}{2}\left[\bar{\Psi}_L^0\gamma^{\mu}\left(\dfrac{g}{\cos\theta_W}Z_{\mu}\right)\Psi_L^0-\bar{\Psi}_L^-\gamma^{\mu}\left(g\left(\dfrac{2\cos^2\theta_W-1}{\cos\theta_W}\right)Z_{\mu}+2eA_{\mu}\right)\Psi_L^-\right]
\nonumber\\
=&\left(\bar{\Psi}_L^0i\slashed{\partial}\Psi_L^0+\bar{\Psi}_L^-i\slashed{\partial}\Psi_L^-\right)
-\dfrac{g}{\sqrt{2}}\left(\bar{\Psi}_L^-\gamma^{\mu}W_{\mu}\Psi_L^0
+\text{h.c}\right)
-\dfrac{g}{2\cos\theta}\bar{\Psi}_L^0\slashed{Z}\Psi_L^0 \nonumber\\
+& g\left(\dfrac{2\cos^2\theta_W-1}{2\cos\theta_W}\right)\bar{\Psi}_L^-\slashed{Z}\Psi_L^--\bar{\Psi}_L^-e\slashed{A}\Psi_L^-\,.
\end{align}
We used the SM relations:
\begin{align}
g\sin\theta_W=g'\cos\theta_W=e \hspace{1.0 cm} \begin{pmatrix}
Z_{\mu} \\ A_{\mu}
\end{pmatrix}=\begin{pmatrix}
\cos\theta_W & -\sin\theta_W \\
\sin\theta_W & \cos\theta_W
\end{pmatrix}\begin{pmatrix}
W^3_{\mu} \\ B_{\mu}
\end{pmatrix}\,.
\end{align}
Analogue, if  $\Psi_{R}^0=(0,{\psi_R^0}^{\dagger})^{T}=P_{R}\Psi_R^0$ and 
$\Psi_{R}^+=(0,{\psi_R^-}^{\dagger})^{T}=P_{R}\Psi_R^+$
\begin{align}
\label{eq:kind}
\bar{\widetilde{R}}_ui\slashed{D}\widetilde{R}_u 
=&\left(\bar{{\Psi}}_R^0i\slashed{\partial}\Psi_R^0+\bar{\Psi}_R^+i\slashed{\partial}\Psi_R^+\right)
+\dfrac{g}{\sqrt{2}}\left(\bar{\Psi}_R^0\gamma^{\mu}W_{\mu}\Psi_R^+
+\text{h.c}\right)
+\dfrac{g}{2\cos\theta}\bar{\Psi}_R^0\slashed{Z}\Psi_R^0 \nonumber\\
-& g\left(\dfrac{2\cos^2\theta_W-1}{2\cos\theta_W}\right)\bar{\Psi}_R^+\slashed{Z}\Psi_R^+-\bar{\Psi}_R^+e\slashed{A}\Psi_R^+\,.
\end{align}
%
Now, replacing the Eq.~\eqref{eq:kinu} and Eq.~\eqref{eq:kind} in the Eq.~\eqref{eq:lint1} we have: 
%
\begin{align}
\label{eq:lint2}
\mathcal{L_{\text{Int}}} =& \dfrac{1}{2}\bar{N}i\slashed{\partial}N 
+\left(\bar{\Psi}_L^0i\slashed{\partial}\Psi_L^0+\bar{\Psi}_L^-i\slashed{\partial}\Psi_L^-+\bar{{\Psi}_R^0}i\slashed{\partial}
\Psi_R^0+\bar{\Psi}_R^+i\slashed{\partial}\Psi_R^+\right)\nonumber\\
-&\dfrac{g}{\sqrt{2}}\left(\bar{\Psi}_L^-\slashed{W}\Psi_L^0-\bar{\Psi}_R^0\slashed{W}\Psi_R^++\text{h.c}\right)
- \dfrac{g}{2\cos\theta}\left(\bar{\Psi}_L^0\slashed{Z}\Psi_L^0-\bar{\Psi}_R^0\slashed{Z}\Psi_R^0\right) \nonumber\\
+& g\left(\dfrac{2\cos^2\theta_W-1}{2\cos\theta_W}\right)\left(\bar{\Psi}_L^-\slashed{Z}\Psi_L^--\bar{\Psi}_R^+\slashed{Z}\Psi_R^+\right)
-\left(\bar{\Psi}_L^-e\slashed{A}\Psi_L^-+\bar{\Psi}_R^+e\slashed{A}\Psi_R^+\right)
\nonumber\\
-& \dfrac{h}{\sqrt{2}}\left(-\lambda_d\bar{N}\Psi_L^0 + \lambda_u\bar{N}\Psi_R^0+\text{h.c}\right)\,.
\end{align}
Therefore, the interaction's Lagrangian is
\begin{align}
\label{eq:lint3}
\mathcal{L}_{\text{Int}} 
=-&\dfrac{g}{\sqrt{2}}\left(\bar{\Psi}_L^-\slashed{W}\Psi_L^0-\bar{\Psi}_R^0\slashed{W}\Psi_R^++\text{h.c}\right)
- \dfrac{g}{2\cos\theta}\left(\bar{\Psi}_L^0\slashed{Z}\Psi_L^0{\color{red}+}\bar{\Psi}_R^0\slashed{Z}\Psi_R^0\right) \nonumber\\
+& g\left(\dfrac{2\cos^2\theta_W-1}{2\cos\theta_W}\right)\left(\bar{\Psi}_L^-\slashed{Z}\Psi_L^-{\color{red}+}\bar{\Psi}_R^+\slashed{Z}\Psi_R^+\right)
-\left(\bar{\Psi}_L^-e\slashed{A}\Psi_L^-+\bar{\Psi}_R^+e\slashed{A}\Psi_R^+\right)
\nonumber\\
-& \dfrac{h}{\sqrt{2}}\left(-\lambda_d\bar{N}\Psi_L^0 + \lambda_u\bar{N}\Psi_R^0+\text{h.c}\right)\,.
\end{align}
%
Finally, we construct the Majorana spinor $X_i^0$ and Dirac spinor $X^{\pm}$  
\begin{align}
X_i^0=\begin{pmatrix}
(\chi_{i}^0)_\alpha \\ (\chi_i^{0\dagger})^{\dot{\alpha}}
\end{pmatrix}
=\begin{pmatrix}
N_{ji}\,\boldsymbol{\Xi}^{0}_j \\
N_{ji}^{\dagger}\,\boldsymbol{{\Xi}^{\dagger}}^{0}_j
\end{pmatrix}
\hspace{1.5 cm}
X^{\pm}=\begin{pmatrix}
\chi^{\pm}_{\alpha} \\ {\chi^{\mp}}^{\dagger\dot{\alpha}}
\end{pmatrix}
=\begin{pmatrix}
\psi_L^{\pm} \\
{\psi_R^{\mp}}^{\dagger}
\end{pmatrix}\,,
\end{align}
where we used the Eq.~\eqref{eq:vecgauge}. Thus, we can know explicitly the Dirac spinor $\Psi_{LR}^0$ and $\Psi_{LR}^{\pm}$ that we constructed in order to get the Eq.~\eqref{eq:lint3}. Those are
\begin{align}
\Psi_L^0=\begin{pmatrix}
\psi_L^0 \\0
\end{pmatrix}
=N_{2i}\begin{pmatrix}
\chi^i_0 \\ 0
\end{pmatrix}
=N_{2i}P_LX_i^0
\hspace{1.0 cm}
\Psi_R^0=\begin{pmatrix}
0 \\ {{\psi}_R^0}^{\dagger}
\end{pmatrix}=N_{3i}\begin{pmatrix}
0 \\ {\chi_i^0}^{\dagger}
\end{pmatrix}=N_{3i}P_RX_i^0\nonumber
\end{align}
%
\begin{align}
\Psi_L^{\pm}=\begin{pmatrix}
\psi_L^{\pm} \\ 0
\end{pmatrix}=P_LX^{\pm}
\hspace{1.0 cm}
\Psi_R^{\pm}=\begin{pmatrix}
0\\ {\psi_R^{\pm}}^{\dagger}
\end{pmatrix}=P_RX^{\pm}\,,
\end{align}
%
Therefore, replacing these spinors in the Eq.~\eqref{eq:lint3}, we get
\begin{align}
\label{eq:lint4}
\mathcal{L}_{\text{Int}}= & 
-\dfrac{g}{\sqrt{2}}(\bar{X}^-\slashed{W}\left(N_{2i}P_L-N_{3i}P_R\right)X_i^0 + \text{h.c})
+ \dfrac{g}{4\cos\theta}\bar{X}_i^0\slashed{Z}\left(N_{2i}N_{2j}-N_{3i}N_{3j}\right)\gamma^5X_j^0 \nonumber\\
+& g\left(\dfrac{2\cos^2\theta_W-1}{2\cos\theta_W}\right)
\bar{X}^-\slashed{Z}X^-
-e\bar{X}^-\slashed{A}X^- 
-\dfrac{1}{\sqrt{2}}h\bar{X}_i^0\left(-\lambda_dN_{2i}N_{1j} + \lambda_uN_{3i}N_{1j}\right)X_j^0\,.
\end{align}
Therefore, the interaction of the DM with de $W$, $Z$ and $h$ SM gauge boson is given by
\begin{align}
\mathcal{L^{\chi}}_{\text{Int}}=-\bar{X}^-\slashed{W}c_{WXX_i}X_i^0
-c_{ZX_iX_j}Z_{\mu}\bar{X}_i^0\gamma^{\mu}\gamma^{5}X_j^0
-c_{hX_iX_j}h\bar{X}_i^0X_j^0
\end{align}
where 
\begin{align}
c_{WXX_i}=& \dfrac{g}{\sqrt{2}}\left(N_{2i}P_L-N_{3i}P_R\right)  \label{eq:cWXXi}\\
c_{ZX_iX_j}=&\frac{g}{4\cos\theta_W}(N_{3i}N_{3j}-N_{2i}N_{2j}) \label{eq:cZXiXj}\\
%problema en el orden
c_{hX_iX_j}=&\frac{1}{\sqrt{2}}(-\lambda_dN_{2i}N_{1j}+\lambda_uN_{3i}N_{1j})\label{eq:cHXiXj}\,.
\end{align}







%%%%%%%%%%%%%%%%%%%%%%%%%%%%%%%%%%%%%% AMPLITUDE SI CROSS SECTION %%%%%%%%%%%%%%%%%%%%%%%%%%%%%%%%%%%%%%%%%
\section{Spin independent cross section in the SDFDM model}
\label{sec:SI-amplitude}

\begin{figure}[h]
  \centering
  \includegraphics[scale=0.45]{SI_N}
  \caption{Scalar inelastic scattering with nucleons.}
  \label{fig:SI_N}
\end{figure}
%
The amplitude of the process shown in the Fig.~\ref{fig:SI_N} is
\begin{align}
\mathcal{M}=&\bar{u}(s_3,p_3)u(s_1,p_1)\dfrac{im_Nf_N}{v}i\Delta_F(p_1-p_3)ic_{hX_1X_1}\bar{u}(s_4,p_4)u(s_2,p_2)\nonumber\\
=& -i\dfrac{im_Nf_N}{v}c_{hX_1X_1}\Delta_F(p_1-p_3) \bar{u}(s_3,p_3)u(s_1,p_1)\bar{u}(s_4,p_4)u(s_2,p_2)\,,
\end{align}
therefore,
\begin{align}
\sum_{s_i}|\mathcal{M}|^2=&\left(\dfrac{m_Nf_N}{v}c_{hX_1X_1}\Delta_F(p_1-p_3)\right)^2
\sum_{s_{1,3}}\left(\bar{u}_{s_3p_3}u_{s_1p_1}\right)\left(\bar{u}_{s_3p_3}u_{s_1p_1}\right)^{\dagger}
\sum_{s_{2,4}}\left(\bar{u}_{s_4p_4}u_{s_2p_2}\right)\left(\bar{u}_{s_4p_4}u_{s_2p_2}\right)^{\dagger}\nonumber\\
=&\left(\dfrac{m_Nf_N}{v}c_{hX_1X_1}\Delta_F(p_1-p_3)\right)^2
\sum_{s_{1,3}}\left(\bar{u}_{s_3p_3}u_{s_1p_1}\bar{u}_{s_1p_1}u_{s_3p_3}\right)
\sum_{s_{2,4}}\left(\bar{u}_{s_4p_4}u_{s_2p_2}\bar{u}_{s_2p_2}u_{s_4p_4}\right)\nonumber\\
=&\left(\dfrac{m_Nf_N}{v}c_{hX_1X_1}\Delta_F(p_1-p_3)\right)^2
\text{Tr}\left[(\slashed{p}_1+m_N)(\slashed{p}_3+m_N)\right]
\text{Tr}\left[(\slashed{p}_2+m_{\chi^0})(\slashed{p}_4+m_{\chi^0})\right]\nonumber\\
=&\left(\dfrac{m_Nf_N}{v}c_{hX_1X_1}\Delta_F(p_1-p_3)\right)^2
\left(4p_1\cdot p_3+4m_N^2\right)\left(4p_2\cdot
p_4+4m_{\chi^0}^2\right)\,.
\end{align}
Now, doing the approximation of $p_i\to0$ we get
\begin{align}
\sum_{s_i}|\mathcal{M}|^2=\overline{|\mathcal{M}|^2}\approx&\left(\dfrac{m_Nf_N}{v}c_{hX_1X_1}\dfrac{1}{m_h^2}\right)^2 16m_N^2m_{\chi^0}^2
=\dfrac{16f_N^2c_{hX_1X_1}^2m_N^4m_{\chi^0}^2}{v^2m_h^4}\,.
\end{align}
Therefore, for an inelastic scattering, the differential cross section is given by
\begin{align}
\dfrac{d\sigma}{d\Omega}=\dfrac{1}{64\pi^2 s}\overline{|\mathcal{M}|^2}
\approx\dfrac{1}{64\pi^2 (m_N+m_{\chi^0}^2)^2}\dfrac{16f_N^2c_{hX_1X_1}^2m_N^4m_{\chi^0}^2}{v^2m_h^4}
=\dfrac{m_r^2}{4\pi^2}\left(\dfrac{c_{hX_1X_1}}{vm_h^2}\right)^2f_N^2m_N^2\,,
\end{align}
where $s=(E_N+E_{\chi^0})^2\approx(m_N+m_{\chi^0})^2$ in the limit of zero momentum and $m_r=\dfrac{m_Nm_{\chi^0}}{(m_N+m_{\chi^0})}$ is the reduced mass of the system.

Finally, integrating in the solid angle, the spin independent cross section a tree level is given by
\begin{align}
\label{eq:SI-tree-level}
\sigma_{SI}=\dfrac{m_r^2}{\pi}\left(\dfrac{c_{hX_1X_1}}{vm_h^2}\right)^2f_N^2m_N^2\,.
\end{align}
% 





%%%%%%%%%%%%%%%%%%%%%%%%%%%%%%%%%%%%%%%%%%%%%%%%%%%%%%%%%%%%%%%%%%%%%%%%%%%%%%%%%%%%%%%%%%%%%%%%%%%%%%%%%%%%
\section{Analytic formulas for masses and mixing matrix of neutral fermions}
\label{sec:analyt-form-mass}

The characteristic equation of the mass matrix~\eqref{eq:Mchi}
is~\cite{Cheung:2013dua}\footnote{The analytic formulas for the neutralino
  masses and the neutralino mixing matrix was
  analyzed in \cite{ElKheishen:1992yv}.}:
\begin{align}
\left[\left({M}^{\chi}_{\text{diag}}\right)_{ii}^2-M_D^2\right]
\left[M_N-\left({M}^{\chi}_{\text{diag}}\right)_{ii}^{\phantom{2}}
\right]
+\tfrac{1}{2}m_{\lambda}^2\left[\left({M}^{\chi}_{\text{diag}}\right)_{ii}+M_D\sin 2\beta
\right]=0\,.
\label{eq:characteristic-equation}
\end{align}
The solutions to the cubic equation in  $\left({M}^{\chi}_{\text{diag}}\right)_{ii}$ are:
\begin{align}
m_1^\chi=&z_2+\dfrac{M_N}{3}\,,&
m_2^\chi=&z_1+\dfrac{M_N}{3}\,, &
m_3^\chi=&z_3+\dfrac{M_N}{3}\,.
\end{align}
where
\begin{align}
z_1&=\left(-\dfrac{q}{2}+\sqrt{\frac{q^2}{4}+\frac{p^3}{27}}\right)^{1/3} + \left(-\dfrac{q}{2}-\sqrt{\dfrac{q^2}{4}+\dfrac{p^3}{27}}\right)^{1/3}\nonumber\\ 
z_2&=-\frac{z_1}{2}+\sqrt{\frac{z_1^2}{4}+\frac{q}{z_1}} \nonumber\\ 
z_3&=-\frac{z_1}{2}-\sqrt{\frac{z_1^2}{4}+\frac{q}{z_1}}\nonumber\\ 
p&=-\frac{1}{3}M_N^2-\left(M_D^2+m_{\lambda}^2\right)  \nonumber\\ 
q&=-\frac{2}{27}M_N^3-\frac{1}{3}M_N\left(M_D^2+m_{\lambda}^2\right)+\left[M_NM_D^2-m_{\lambda}^2\sin(2\beta) M_D\right].
\end{align}

Notice that ${q^2}/{4}+{p^3}/{27} < 0$ and therefore, we have three real masses $m_i^\chi$ $(i=1,2,3)$.

Expanding the eigensystem in eq.~\eqref{eq:chidiag} by  assuming  that ${N}_{1i}\neq 0$, we have 

\begin{align*}
{M}^{\chi}_{21}\frac{{N}_{2i}}{{N}_{1i}}+{M}^{\chi}_{31}\frac{{N}_{3i}}{{N}_{1i}}&=-({M}^{\chi}_{11}-m_i^\chi)\nonumber \\
({M}^{\chi}_{22}-m_i^\chi)\frac{{N}_{2i}}{{N}_{1i}}+{M}^{\chi}_{32}\frac{{N}_{3i}}{{N}_{1i}}&=-{M}^{\chi}_{12} \nonumber\\
{M}^{\chi}_{23}\frac{{N}_{2i}}{{N}_{1i}}+({M}^{\chi}_{33}-m_i^\chi)\frac{{N}_{3i}}{{N}_{1i}}&=-{M}^{\chi}_{13}\,,
\end{align*}
where
\begin{align}
\label{eq:N1i}
{N}_{1i}=\left[1+\left(\frac{{N}_{2i}}{{N}_{1i}}\right)^2+\left(\frac{{N}_{3i}}{{N}_{1i}}\right)^2\right]^{-1/2}.
\end{align}

Using the matrix $\textbf{M}^{\chi}$ given in the eq. \eqref{eq:Mchi}, we get the ratios

\begin{align}
\label{eq:exNmn}
\frac{{N}_{2i}}{{N}_{1i}}
&=-\frac{m_{\lambda} \cos\beta}{m_i^\chi}+\frac{M_D}{m_i^\chi}\frac{[m_i^\chi(M_N-m_i^\chi)+m_{\lambda}^2\cos\beta^2 ]}{m_{\lambda}(m_i^\chi\sin\beta +M_D\cos\beta )}\,, \nonumber\\  
\frac{{N}_{3i}}{{N}_{1i}}
&=-\frac{[m_i^\chi(M_N-m_i^\chi)+m_{\lambda}^2 \cos\beta^2]}{m_{\lambda} (m_i^\chi\sin\beta+M_D\cos\beta)}.
\end{align}


\subsection{Approximate mixing matrix}
By using the analytical expressions for the mixing ratios of
eq.~\eqref{eq:exNmn} with the approximate eigenvalues~\eqref{eq:ml2}
in eq.~\eqref{eq:N1i}, we obtain

\begin{align}
N_{11}^2=& 1-\frac{\left[M_D^2+M_N^2+2M_DM_N\sin(2\beta)\right]m^2_{\lambda}}{(M_D^2-M_N^2)^2}+\mathcal{O}\left( m_{\lambda}^4 \right)\nonumber\\
N_{12}^2 =&   \frac{[\sin (2 \beta )+1] m_{\lambda }^2}{2 \left(M_N-M_D\right)^2}+\mathcal{O}\left( m_{\lambda}^4 \right)\nonumber\\
N_{13}^2=&  -\frac{[\sin (2 \beta )-1] m_{\lambda }^2}{2 \left(M_D+M_N\right)^2}+\mathcal{O}\left( m_{\lambda}^4 \right)\,.
\end{align}
\begin{align}
N_{21}^2=&\frac{m_{\lambda }^2 \left(\sin\beta  M_D+\cos\beta
   M_N\right)^2}{\left(M_N^2-M_D^2\right)^2}+\mathcal{O}\left( m_{\lambda}^4 \right)\nonumber\\
N_{22}^2=&\frac{1}{2}-\frac{m_{\lambda }^2 (\sin\beta+\cos\beta) \left[\cos\beta M_N-\sin\beta  \left(M_N-2 M_D\right)\right]}{4 M_D \left(M_N-M_D\right)^2}+\mathcal{O}\left( m_{\lambda}^4 \right)\nonumber\\
   N_{23}^2=&\frac{1}{2}+\frac{m_{\lambda }^2 (\cos\beta-\sin\beta) \left[\sin\beta \left(2
   M_D+M_N\right)+\cos\beta M_N\right]}{4 M_D \left(M_D+M_N\right)^2}+\mathcal{O}\left( m_{\lambda}^4 \right).
\end{align}

\begin{align}
\label{eq:mixl2}
N_{31}^2=&\left(\frac{M_{D} \cos\beta + M_{N} \sin\beta}{M_{N}^{2}- M_{D}^{2}} \right)^{2}   m_{\lambda}^{2}
+\mathcal{O}\left( m_{\lambda}^4 \right)\nonumber\\
N_{32}^2=&\frac{1}{2}-\frac{\left[M_{N} \sin\beta - \left( M_{N}- 2 M_{D}\right) \cos\beta\right] \left(\cos\beta+\sin\beta \right)}{4 M_{D} \left(M_{N}- M_{D}\right)^{2}}\,m_{\lambda}^{2}+\mathcal{O}\left( m_{\lambda}^4 \right)  \nonumber\\
N_{33}^2=&\frac{1}{2}- \frac{\left[M_{N} \sin\beta + \left(M_{N} + 2 M_{D}\right) \cos\beta\right] \left(\cos\beta- \sin\beta \right)}{4 M_{D} \left(M_{N} + M_{D} \right)^{2}} \,m_{\lambda}^{2}+\mathcal{O}\left( m_{\lambda}^4 \right).
\end{align}
In particular, with eq.~\eqref{eq:ml2} and the expressions for $N_{3i}^2$, the identity \eqref{eq:divcan}
is satisfied up to terms of order $\mathcal{O}\left( m_{\lambda}^4
\right)$.






%%%%%%%%%%%%%%%%%%%%%%%%%%%%%%%%%%%%%%%%%%%%%%%%%%%%%%%%%%%%%%%%%%%%%%%%%%%%%%%%%%%%%%%%%%%%%%%%%%%%%%%%%%%%
\section{One-loop neutrino masses in the gauge base}
\label{sec:mass-interaction-basis}

By using the Feynman rules for Weyl spinors we have the diagrams for
neutrino masses at 1-loop shown in \ref{fig:t13aweyl}
%
\begin{figure}[h]
  \centering
\includegraphics[scale=0.3]{T13AweylR}\qquad \includegraphics[scale=0.3]{T13AweylRL}
  \caption{1-loop neutrino mass in the interaction basis}
  \label{fig:t13aweyl}
\end{figure}

The results from \cite{Bonnet:2012kz,Suematsu:2010nd} adapted to our case imply that ($\lambda_d \to Y_L\; \lambda_u \to Y_R$)
%check nuint.pdf
\begin{align}
  M^{\nu}_{ij}=2\sum_{\alpha}h_{i\alpha}h_{j\alpha} &\left[  \lambda_u^2\int \frac{d^4 k}{(2\pi)^4}i S_F \left(k,M_D \right)P_R S_F \left(k,M_N \right)P_R i S_F \left(k,M_D \right)
             \Delta_F \left( k+p,m_{S_\alpha} \right)\right. \nonumber\\
          &\left.+\lambda_d^2\int \frac{d^4 k}{(2\pi)^4}i S_F \left(k,M_D \right)P_L S_F \left(k,M_N \right)P_L i S_F \left(k,M_D \right)
             \Delta_F \left( k+p,m_{S_\alpha} \right) 
            \right].
\end{align}

\begin{align}
  M^{\nu}_{ij}=\sum_{\alpha}\frac{2h_{i\alpha}h_{j\alpha}v^2 M_N}{(4\pi)^2} 
      \left[\lambda_u^{2} \, J_4\left(M_D^2,M_D^2,M_N^2,m^{S}_{\alpha}  \right)
            + \lambda_d^2 M_D^2  \, I_4\left(M_D^2,M_D^2,M_N^2,m^2_{S_\alpha}  \right)
      \right],
\end{align}
where
\begin{align}
   J_4\left(M_D^2,M_D^2,M_N^2,m^2_{S}  \right)=&\frac{(4\pi)^2}{i}
   \int \frac{d^4 k}{(2\pi)^4}
\frac{k^2}{\left(k^2-m_D^2\right)^2\left( k^2-M_N^2 \right)\left(k^2-m_{S_\alpha}^2\right)} \nonumber\\
   I_4\left(M_D^2,M_D^2,M_N^2,m^2_{S}  \right)=&\frac{(4\pi)^2}{i}
   \int \frac{d^4 k}{(2\pi)^4}
\frac{1}{\left(k^2-m_D^2\right)^2\left( k^2-M_N^2 \right)\left(k^2-m_{S_\alpha}^2\right)}.
\end{align}
After integration
%see calculoij.pdf
\begin{align*}
  J_4\left(M_D^2,M_D^2,M_N^2,m^2_{S}  \right)=
-&\left\{  \left[\frac{M_D^2}{\left(M_N^2-M_D^2\right)\left(m_{S}^2-M_D^2\right)} 
-\frac{M_D^4 \left(2M_D^2-M_N^2-m_{S_\alpha}^2\right)}{2\left(M_N^2-M_D^2\right)^2\left(m_{S}^2-M_D^2\right)^2} \right]\ln M_D^2\right.\nonumber\\
&-\frac{M_N^4\ln M_N^2}{2\left( m_{S_\alpha}^2-M_N^2 \right)\left( M_D^2-M_N^2 \right)^2}-
\frac{m_{S_\alpha}^4\ln m_{S_\alpha}^2}{2\left( m_N^2-m_{S_\alpha}^2 \right)\left( M_D^2-m_{S_\alpha}^2 \right)^2}
\nonumber\\
 &\left.+\frac{M_D^2}{2\left(M_N^2-M_D^2  \right)\left(m_{S_\alpha}^2-M_D^2  \right)} \right\},
\end{align*}
\begin{align}
      I_4&\left(M_D^2,M_D^2,M_N^2,m^2_{S}  \right)=\nonumber\\
&\left[\frac{M_N^2\ln \left( M_N^2/M_D^2 \right)}{\left( M_N^2-M_D^2 \right)^4\left( M_N^2-m_{S_\alpha}^2 \right)^2}
      -\frac{m_{S_\alpha}^2\ln \left( m_{S_\alpha}^2/M_D^2 \right)}{\left( m_{S_\alpha}^2-M_D^2 \right)^4\left( M_N^2-m_{S_\alpha}^2 \right)^2}-
      \frac{1}{\left(m_{S_\alpha}^2-M_D^2\right)^2\left(M_N^2 -M_D^2 \right)^2} 
 \right],
\end{align}
or
\begin{align}
   I_4\left(M_D^2,M_D^2,M_N^2,m^2_{S}  \right)=
-&\left[ \frac{\left(M_D^4-M_N^2 m_{S_\alpha}^2\right)\ln M_D^2}{\left(M_N^2 -M_D^2 \right)^2\left(m_{S_\alpha}^2 -M_D^2 \right)^2} 
+\frac{M_N^2 \ln M_N^2}{\left(m_{S_\alpha}^2 -M_N^2 \right)\left(M_D^2-M_N^2\right)^2}\right. \nonumber\\
&\left.+\frac{m_{S_\alpha}^2 \ln m_{S_\alpha}^2}{\left(M_N^2 -m_{S_\alpha}^2 \right)\left(M_D^2-m_{S_\alpha}^2\right)^2} 
-\frac{1}{\left(m_{S_\alpha}^2-M_D^2\right)\left(M_N^2 -M_D^2 \right)} 
\right].
\end{align}

In  \cite{Fraser:2014yha} only the contribution proportional to $\lambda_u$ is considered in the limit $M_N\to 0$\footnote{$\lim_{x\to 0}x\ln x=\lim_{x\to 0}\dfrac{\ln x}{1/x}=0$}, 
\begin{align*}
   J_4\left(M_D^2,M_D^2,M_N^2,m^2_{S}  \right)=&
-\left\{  \left[-\frac{M_D^2}{M_D^2\left(m_{S}^2-M_D^2\right)} 
-\frac{M_D^4 \left(2M_D^2-m_{S_\alpha}^2\right)}{2M_D^4\left(m_{S}^2-M_D^2\right)^2} \right]\ln M_D^2\right.\nonumber\\
&\qquad+\frac{m_{S_\alpha}^4\ln m_{S_\alpha}^2}{2m_{S_\alpha}^2\left( M_D^2-m_{S_\alpha}^2 \right)^2} 
\left.-\frac{M_D^2}{2M_D^2\left(m_{S_\alpha}^2-M_D^2  \right)} \right\}\nonumber\\
=&  -\left\{  -\left[\frac{1}{\left(m_{S}^2-M_D^2\right)} 
+\frac{\left(2M_D^2-m_{S_\alpha}^2\right)}{2\left(m_{S}^2-M_D^2\right)^2} \right]\ln M_D^2
+\frac{m_{S_\alpha}^2\ln m_{S_\alpha}^2}{2\left( M_D^2-m_{S_\alpha}^2 \right)^2} 
-\frac{1}{2\left(m_{S_\alpha}^2-M_D^2  \right)} \right\}\nonumber\\
=&  -\frac{1}{\left(m_{S}^2-M_D^2\right)}\left\{  -\left[1 
+\frac{\left(2M_D^2-m_{S_\alpha}^2\right)}{2\left(m_{S}^2-M_D^2\right)} \right]\ln M_D^2
+\frac{m_{S_\alpha}^2\ln m_{S_\alpha}^2}{2\left( M_D^2-m_{S_\alpha}^2 \right)} 
-\frac{1}{2} \right\}\nonumber\\
= & \frac{1}{2\left(m_{S}^2-M_D^2\right)}\left\{  \frac{m_{S_\alpha}^2\ln M_D^2}{\left(m_{S}^2-M_D^2\right)}
-\frac{m_{S_\alpha}^2\ln m_{S_\alpha}^2}{\left( M_D^2-m_{S_\alpha}^2 \right)} +1 \right\}\nonumber\\
=&  \frac{1}{2\left(m_{S}^2-M_D^2\right)}\left[ 1+ \frac{m_{S_\alpha}^2\ln \left( M_D^2/m_{S_\alpha}^2 \right)}{\left(m_{S}^2-M_D^2\right)}
\right]\nonumber\\
=& - \frac{1}{2\left(M_D^2-m_{S}^2\right)}\left[ 1-\frac{m_{S_\alpha}^2\ln \left( M_D^2/m_{S_\alpha}^2 \right)}{\left(M_D^2-m_{S}^2\right)}
\right].
\end{align*}
So that
\begin{align*}
M^{\nu}_{ij}=-\sum_{\alpha}\frac{h_{i\alpha}h_{j\alpha}\left(\lambda_u^2 v^2\right)M_N}{16\pi^2\left(M_D^2-m_{S_\alpha}^2\right)}
\left[ 1-\frac{m_{S_\alpha}^2\ln \left( M_D^2/m_{S_\alpha}^2 \right)}{\left(M_D^2-m_{S_\alpha}^2\right)}\right].
\end{align*}






%%%%%%%%%%%%%%%%%%%%%%%%%%%%%%%%%%%%%%%%%%%%%%%%%%%%%%%%%%%%%%%%%%%%%%%%%%%%%%%%%%%%%%%%%%%%%%%%%%%%%%%%%
\section{$\mu  \rightarrow e \gamma$ process in the T13A model}
\label{sec:Ap-muegamma}

A loop computation of $\mu  \rightarrow e \gamma$ into our model has account the coupling to the photon with the new charge fermions $\chi^{\pm}$ and the coupling into the SM leptons, the new charge fermions and the new  neutral scalars.
The three diagrams important for this computations are shown in the Fig.~\ref{fig:muegamma}.
 
The radiative decay $f_1(p_1)  \rightarrow f_2(p_2) \gamma (q)$ where $q=p_1-p_2$, $f_1$ has a mass $m_1$ and $f_2 $ has a mass $m_2$,is compute using~\cite{Lavoura:2003xp}.
We take the fermions on shell: $p_1^2=m_1^2$  y $p_2^2=m_2^2$. They are represented by spinors $u_1$ and $u_2$, which satisfy $\cancel{p}_1u_1=m_1u_1$ and $\bar{u}_2\cancel{p}_2=m_2\bar{u}_2$.

The amplitude for the decay is $e \epsilon^*_{\mu}(q)M^{\mu}$, where $\epsilon^*_{\mu}(q)$ is the outgoing photon polarization and $e$ is the electric charge of the positron. We know that the gauge invariance implies that $q_{\mu}M^{\mu}$ must be zero, therefore $M^{\mu}$ has the form:
\begin{align}
M^{\mu}=\bar{u}_2(p_2)( \sigma_L\Sigma_L^{\mu}+\sigma_R\Sigma_R^{\mu} +\mathcal{O}^{\mu} )u_1(p_1)\,,
\end{align}
where:
\begin{align}
\Sigma_L^{\mu}=&(p_1^{\mu}+p_2^{\mu})P_L-\gamma^{\mu}(m_2P_L+m_1P_R)\\	
\Sigma_R^{\mu}=&(p_1^{\mu}+p_2^{\mu})P_R\gamma^{\mu}(m_2P_R+m_1P_L)\,.
\end{align}
%
We only take care of $\sigma_{L}$ and $\sigma_{R}$ factors, not for $\mathcal{O}^{\mu}$, because they are the only relevant parameters important for $f_1 \rightarrow f_2 \gamma$ decay and on-shell photon. Therefore, we will not put the $\mathcal{O}^{\mu}$ term anymore.

Alternatively, we can write $M^{\mu}$ in the form:
\begin{align}
M^{\mu}=\bar{u}_2(p_2)( i\sigma^{\mu\nu}q_{\nu}(\sigma_LP_L+\sigma_RP_R))u_1(p_1)\,,
\end{align}
%
where in general, the partial width for $f_1\rightarrow f_2\gamma$ is given by:
%
\begin{align}
\Gamma = \dfrac{(m_1^2-m_2^2)^3(|\sigma_L|^2+|\sigma_R|^2)}{16\pi m_1^3}\,.
\label{partial_width_muegamma}
\end{align}

With all this in mind, our work is determinate the $\sigma_L$ and the $\sigma_R$ in T13A model and use the Eq.~\eqref{partial_width_muegamma}.

\subsection{$\sigma_L$ and $\sigma_R$ in T13A}

In general for a Yukawa interaction to the form:
\begin{align}
\label{eq:general-yukawa-lagrangian}
\mathcal{L}_{\text{Yukawa}}=B\overline{F}(L_iP_L+R_iP_R)f_i+B^*\overline{f}_i(L_i^*P_R+R_i^*P_L)F\,,
\end{align}
where the fermions $f_i$ have charge $Q_f=e$, the spin-$0$ boson $B$ have charge $Q_B$ and the fermions $f_i$ have interaction with the boson $B$ and spin-$1/2$ fermion $F$ with arbitrary dimensionless numerical coefitients $L_1,L_2$, $R_1$ and $R_2$, the $\sigma_L$ and $\sigma_R$ are given by:
\begin{align}
\sigma_L=& Q_f(\rho k_1 +\lambda k_2 + v k_3)+Q_B(\rho \overline{k}_1 + \lambda \overline{k}_2 + v \overline{k}_3)\\
\sigma_R=& Q_f(\lambda k_1 +\rho k_2 + \zeta k_3)+Q_B(\lambda \overline{k}_1 + \rho \overline{k}_2 + \zeta \overline{k}_3)\,,
\end{align} 
with:
\begin{align}
\lambda=&L_2^+L_1\\
\rho=&R_2^*R_1\\
\zeta=&L_2^*R_1\\
v=&R_2^*L_1\\
k_1=&m_1(c_1+d_1+f)\\
k_2=&m_2(c_2+d_2+f)\\
k_3=&m_F(c_1+c_2)\,.
\end{align}
There are others that are not important because in the T13A model $Q_B=0$.

In our model we have the next term in the Yukawa lagrangian:
\begin{align}
\mathcal{L}&\supset h_{i\alpha}\overline{e_{Li}}\psi^-_{R}S_\alpha+ \text{h.c} =S_\alpha\overline{\psi^-}[h_{i\alpha}P_L]e_i + S_\alpha\overline{e_i}[h_{i\alpha}P_R]\psi^- \,,
\end{align}
%
therefore, we can do the next identification with the Lagrangian~\ref{eq:general-yukawa-lagrangian}. We got
%
\begin{center}
\begin{tabular}{|c|c|c|}
\hline
B $\rightarrow S_\alpha$ & $R_i \rightarrow  0$ & $R_i^* \rightarrow  h_{i\alpha}$  \\
F $\rightarrow \psi^-$ & $L_i \rightarrow h_{i\alpha}$ & $L_i^* \rightarrow  0$ \\
$f_i \rightarrow e_{i}$ &  & \\ 
\hline
\end{tabular}
\end{center}
and
\begin{align}
\sigma_L=&eh_{2\alpha}h_{1\alpha}m_1(c_1+d_1+f)=eh_{2\alpha}h_{1\alpha}m_1(c_1+\dfrac{3}{2}d) \nonumber \\
\sigma_R=&eh_{2\alpha}h_{1\alpha}m_2(c_2+d_2+f)=eh_{2\alpha}h_{1\alpha}m_2(c_2+\dfrac{3}{2}d) \approx 0 \,.
\label{sigmas}
\end{align}
In the last equation, I used the relation $d_1=d_2=2f=d$ and $m_1=m_{\mu} \gg m_2=m_{e} $.

According with \cite{Lavoura:2003xp}, the parameter $c_i=c,d_i=d$ and $f$ are related with the loop integrals. Particularly we have:
\begin{align}
\bigg(c+\dfrac{3}{2}d\bigg)=&\dfrac{i}{16\pi^2 m_B^2}\bigg[\dfrac{x^2-5x-2}{12(x-1)^3}+\dfrac{x \ln x}{2(x-1)^4}\bigg]=\dfrac{i}{16\pi^2 m_B^2}\dfrac{1}{2}\bigg[\dfrac{x^3-6x^2+3x+2+6x\ln x}{6(x-1)^4}\bigg]\,,
\end{align}
where $x=m_F^2/m_{\phi}^2 $. 
%
Finally, putting all this in the Eq.~\eqref{sigmas} we have:
\begin{align}
\sigma_L=&\dfrac{i eh_{2\alpha}h_{1\alpha}m_{\mu}}{16\pi^2 m_B^2}\dfrac{1}{2}F(x)\nonumber \\
\sigma_R \approx & 0 \,,
\label{sigmas_result}
\end{align}
where: 
\begin{align}
F(x)=\bigg[\dfrac{x^3-6x^2+3x+2+6x\ln x}{6(x-1)^4}\bigg] \hspace{1 cm} \text{with} \hspace{1 cm} x=\dfrac{M_D^2}{m_{S_{\alpha}}^2}\,.
\label{Fx}
\end{align}

\subsection{ Branching ($\mu \rightarrow e \gamma$)}
Putting the Eq.~\eqref{sigmas_result} into the Eq.~\eqref{partial_width_muegamma} we have:
\begin{align}
\Gamma =&\dfrac{(m_{\mu}^2-m_e^2)^3}{16\pi m_{\mu}^3}(|\sigma_L|^2) \approx \dfrac{m_{\mu}^3 |\sigma_L|^2}{16\pi}
= \dfrac{m_{\mu}^5}{16\pi}\bigg| \dfrac{e h_{1\alpha}h_{2\alpha}}{16 \pi^2 m_{S_\alpha}^2} \dfrac{F(x)}{2} \bigg|^2\,,
\end{align}
and therefore:
\begin{align}
\operatorname{Br}(\mu \rightarrow e \gamma) 
=& \dfrac{\Gamma(\mu \rightarrow e \gamma )}{\Gamma(\mu \rightarrow e \bar{\nu}_e \nu_{\mu})}
=\dfrac{\dfrac{m_{\mu}^5}{16\pi}\bigg| \dfrac{e h_{1\alpha}h_{2\alpha}}{16 \pi^2 m_{S_\alpha}^2} \dfrac{F(x)}{2} \bigg|^2}{ \dfrac{G_F^2 m_{\mu}^5}{192\pi^3} }
=\dfrac{3}{4}\dfrac{1}{G_F^2}\bigg(\dfrac{e^2}{4\pi}\bigg)\dfrac{1}{16\pi}\bigg|\dfrac{F(x)h_{1\alpha}h_{2\alpha}}{m_{S_\alpha}^2}\bigg|^2 \nonumber \\
=&\dfrac{3}{4}\dfrac{\alpha_{em}}{16 \pi G_F^2}\left|\sum_{\alpha}\dfrac{F(x)h_{1\alpha}h_{2\alpha}}{m_{S_\alpha}^2}  \right|^2 \,.
%\label{branching_muegamma}
\end{align}
%
In general, for a complex Yukawa coupling $h_{i\alpha}$ we have,
\begin{align}
\operatorname{Br}(\mu \rightarrow e \gamma)=&\dfrac{3}{4}\dfrac{\alpha_{\text{em}}}{16 \pi G_F^2}\left|\sum_{\alpha}
h_{1\alpha}\frac{F\left(M_D^2/m_{S_{\alpha}}^2  \right) }{m_{S_\alpha}^2}h_{2\alpha}^{*}  \right|^2 \,.
\end{align}





\newpage

%%%%%%%%%%%%%%%%%%%%%%%%%%%%%%%%%%%%%%%%%%%%%%%%%%%%%%%%%%%%%%%%%%%%%%%%%%%%%%%%%%%%%%%%%%%%%%%%%%%%%%%%%
\section{$\chi^0\chi^0\to\gamma Z$  in the SDFDM model}
\label{sec:xxtogz}

\subsection{Diagrams for the process $\chi^0\chi^0\to\gamma Z$  in the SDFDM model}
\label{sec:xx-to-gz-diagrams}

%
%\begin{figure}
%\centering
%\includegraphics[scale=0.57]{2F-gz-diagrams}
%\caption{Feynman diagrams for $\chi^0\chi^0 \to\gamma Z$ generated with \textsc{FeynArts}~\cite{Hahn:2000kx}. In order to be more readable we show the diagrams in the unitary gauge. We only used the third family of quarks and one lepton family $f$.  We don't show the equivalent topologies with crossed initial or final state legs.}
%\label{fig:2F-to-GZ}
%\end{figure}
%%

\begin{figure}[h]
\centering
\includegraphics[scale=0.5]{gz1}
\end{figure}
%
\begin{figure}[h]
\centering
\includegraphics[scale=0.5]{gz2}
\caption{Feynman diagrams for $\chi^0\chi^0 \to\gamma Z$ generated with \textsc{FeynArts}~\cite{Hahn:2000kx}. In order to be more readable we show the diagrams in the unitary gauge. We only used the third family of quarks and one lepton family $\tau= $ ta.  We don't show the equivalent topologies with crossed initial or final state legs.}
\label{fig:2F-to-GZ}
\end{figure}





