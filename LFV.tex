%%%%%%%%%%%%%%%%%%%%%%%%%%%%%%%%%%%%%%%%%%%%%%%%%%%%%%%%%%%%%%%%%%%%%%%%%%%%%%%%%%%%%%%%%%%%%%%
\section{$\mu  \rightarrow e \gamma$ process in the T13A model}
\label{sec:Ap-muegamma}

A loop computation of $\mu  \rightarrow e \gamma$ into our model has account the coupling to the photon with the new charge fermions $\chi^{\pm}$ and the coupling into the SM leptons, the new charge fermions and the new  neutral scalars.
The three diagrams important for this computations are shown in the Fig.~\ref{fig:muegamma}.
 
The radiative decay $f_1(p_1)  \rightarrow f_2(p_2) \gamma (q)$ where $q=p_1-p_2$, $f_1$ has a mass $m_1$ and $f_2 $ has a mass $m_2$,is compute using~\cite{Lavoura:2003xp}.
We take the fermions on shell: $p_1^2=m_1^2$  y $p_2^2=m_2^2$. They are represented by spinors $u_1$ and $u_2$, which satisfy $\cancel{p}_1u_1=m_1u_1$ and $\bar{u}_2\cancel{p}_2=m_2\bar{u}_2$.

The amplitude for the decay is $e \epsilon^*_{\mu}(q)M^{\mu}$, where $\epsilon^*_{\mu}(q)$ is the outgoing photon polarization and $e$ is the electric charge of the positron. We know that the gauge invariance implies that $q_{\mu}M^{\mu}$ must be zero, therefore $M^{\mu}$ has the form:
\begin{align}
M^{\mu}=\bar{u}_2(p_2)( \sigma_L\Sigma_L^{\mu}+\sigma_R\Sigma_R^{\mu} +\mathcal{O}^{\mu} )u_1(p_1)\,,
\end{align}
where:
\begin{align}
\Sigma_L^{\mu}=&(p_1^{\mu}+p_2^{\mu})P_L-\gamma^{\mu}(m_2P_L+m_1P_R)\\	
\Sigma_R^{\mu}=&(p_1^{\mu}+p_2^{\mu})P_R\gamma^{\mu}(m_2P_R+m_1P_L)\,.
\end{align}
%
We only take care of $\sigma_{L}$ and $\sigma_{R}$ factors, not for $\mathcal{O}^{\mu}$, because they are the only relevant parameters important for $f_1 \rightarrow f_2 \gamma$ decay and on-shell photon. Therefore, we will not put the $\mathcal{O}^{\mu}$ term anymore.

Alternatively, we can write $M^{\mu}$ in the form:
\begin{align}
M^{\mu}=\bar{u}_2(p_2)( i\sigma^{\mu\nu}q_{\nu}(\sigma_LP_L+\sigma_RP_R))u_1(p_1)\,,
\end{align}
%
where in general, the partial width for $f_1\rightarrow f_2\gamma$ is given by:
%
\begin{align}
\Gamma = \dfrac{(m_1^2-m_2^2)^3(|\sigma_L|^2+|\sigma_R|^2)}{16\pi m_1^3}\,.
\label{partial_width_muegamma}
\end{align}

With all this in mind, our work is determinate the $\sigma_L$ and the $\sigma_R$ in T13A model and use the Eq.~\eqref{partial_width_muegamma}.

\section{$\sigma_L$ and $\sigma_R$ in T13A}

In general for a Yukawa interaction to the form:
\begin{align}
\label{eq:general-yukawa-lagrangian}
\mathcal{L}_{\text{Yukawa}}=B\overline{F}(L_iP_L+R_iP_R)f_i+B^*\overline{f}_i(L_i^*P_R+R_i^*P_L)F\,,
\end{align}
where the fermions $f_i$ have charge $Q_f=e$, the spin-$0$ boson $B$ have charge $Q_B$ and the fermions $f_i$ have interaction with the boson $B$ and spin-$1/2$ fermion $F$ with arbitrary dimensionless numerical coefitients $L_1,L_2$, $R_1$ and $R_2$, the $\sigma_L$ and $\sigma_R$ are given by:
\begin{align}
\sigma_L=& Q_f(\rho k_1 +\lambda k_2 + v k_3)+Q_B(\rho \overline{k}_1 + \lambda \overline{k}_2 + v \overline{k}_3)\\
\sigma_R=& Q_f(\lambda k_1 +\rho k_2 + \zeta k_3)+Q_B(\lambda \overline{k}_1 + \rho \overline{k}_2 + \zeta \overline{k}_3)\,,
\end{align} 
with:
\begin{align}
\lambda=&L_2^+L_1\\
\rho=&R_2^*R_1\\
\zeta=&L_2^*R_1\\
v=&R_2^*L_1\\
k_1=&m_1(c_1+d_1+f)\\
k_2=&m_2(c_2+d_2+f)\\
k_3=&m_F(c_1+c_2)\,.
\end{align}
There are others that are not important because in the T13A model $Q_B=0$.

In our model we have the next term in the Yukawa lagrangian:
\begin{align}
\mathcal{L}&\supset h_{i\alpha}\overline{e_{Li}}\psi^-_{R}S_\alpha+ \text{h.c} =S_\alpha\overline{\psi^-}[h_{i\alpha}P_L]e_i + S_\alpha\overline{e_i}[h_{i\alpha}P_R]\psi^- \,,
\end{align}
%
therefore, we can do the next identification with the Lagrangian~\ref{eq:general-yukawa-lagrangian}. We got
%
\begin{center}
\begin{tabular}{|c|c|c|}
\hline
B $\rightarrow S_\alpha$ & $R_i \rightarrow  0$ & $R_i^* \rightarrow  h_{i\alpha}$  \\
F $\rightarrow \psi^-$ & $L_i \rightarrow h_{i\alpha}$ & $L_i^* \rightarrow  0$ \\
$f_i \rightarrow e_{i}$ &  & \\ 
\hline
\end{tabular}
\end{center}
and
\begin{align}
\sigma_L=&eh_{2\alpha}h_{1\alpha}m_1(c_1+d_1+f)=eh_{2\alpha}h_{1\alpha}m_1(c_1+\dfrac{3}{2}d) \nonumber \\
\sigma_R=&eh_{2\alpha}h_{1\alpha}m_2(c_2+d_2+f)=eh_{2\alpha}h_{1\alpha}m_2(c_2+\dfrac{3}{2}d) \approx 0 \,.
\label{sigmas}
\end{align}
In the last equation, I used the relation $d_1=d_2=2f=d$ and $m_1=m_{\mu} \gg m_2=m_{e} $.

According with \cite{Lavoura:2003xp}, the parameter $c_i=c,d_i=d$ and $f$ are related with the loop integrals. Particularly we have:
\begin{align}
\bigg(c+\dfrac{3}{2}d\bigg)=&\dfrac{i}{16\pi^2 m_B^2}\bigg[\dfrac{x^2-5x-2}{12(x-1)^3}+\dfrac{x \ln x}{2(x-1)^4}\bigg]=\dfrac{i}{16\pi^2 m_B^2}\dfrac{1}{2}\bigg[\dfrac{x^3-6x^2+3x+2+6x\ln x}{6(x-1)^4}\bigg]\,,
\end{align}
where $x=m_F^2/m_{\phi}^2 $. 
%
Finally, putting all this in the Eq.~\eqref{sigmas} we have:
\begin{align}
\sigma_L=&\dfrac{i eh_{2\alpha}h_{1\alpha}m_{\mu}}{16\pi^2 m_B^2}\dfrac{1}{2}F(x)\nonumber \\
\sigma_R \approx & 0 \,,
\label{sigmas_result}
\end{align}
where: 
\begin{align}
F(x)=\bigg[\dfrac{x^3-6x^2+3x+2+6x\ln x}{6(x-1)^4}\bigg] \hspace{1 cm} \text{with} \hspace{1 cm} x=\dfrac{M_D^2}{m_{S_{\alpha}}^2}\,.
\label{Fx}
\end{align}

\section{ Branching ($\mu \rightarrow e \gamma$)}
Putting the Eq.~\eqref{sigmas_result} into the Eq.~\eqref{partial_width_muegamma} we have:
\begin{align}
\Gamma =&\dfrac{(m_{\mu}^2-m_e^2)^3}{16\pi m_{\mu}^3}(|\sigma_L|^2) \approx \dfrac{m_{\mu}^3 |\sigma_L|^2}{16\pi}
= \dfrac{m_{\mu}^5}{16\pi}\bigg| \dfrac{e h_{1\alpha}h_{2\alpha}}{16 \pi^2 m_{S_\alpha}^2} \dfrac{F(x)}{2} \bigg|^2\,,
\end{align}
and therefore:
\begin{align}
\operatorname{Br}(\mu \rightarrow e \gamma) 
=& \dfrac{\Gamma(\mu \rightarrow e \gamma )}{\Gamma(\mu \rightarrow e \bar{\nu}_e \nu_{\mu})}
=\dfrac{\dfrac{m_{\mu}^5}{16\pi}\bigg| \dfrac{e h_{1\alpha}h_{2\alpha}}{16 \pi^2 m_{S_\alpha}^2} \dfrac{F(x)}{2} \bigg|^2}{ \dfrac{G_F^2 m_{\mu}^5}{192\pi^3} }
=\dfrac{3}{4}\dfrac{1}{G_F^2}\bigg(\dfrac{e^2}{4\pi}\bigg)\dfrac{1}{16\pi}\bigg|\dfrac{F(x)h_{1\alpha}h_{2\alpha}}{m_{S_\alpha}^2}\bigg|^2 \nonumber \\
=&\dfrac{3}{4}\dfrac{\alpha_{em}}{16 \pi G_F^2}\left|\sum_{\alpha}\dfrac{F(x)h_{1\alpha}h_{2\alpha}}{m_{S_\alpha}^2}  \right|^2 \,.
%\label{branching_muegamma}
\end{align}
%
In general, for a complex Yukawa coupling $h_{i\alpha}$ we have,
\begin{align}
\operatorname{Br}(\mu \rightarrow e \gamma)=&\dfrac{3}{4}\dfrac{\alpha_{\text{em}}}{16 \pi G_F^2}\left|\sum_{\alpha}
h_{1\alpha}\frac{F\left(M_D^2/m_{S_{\alpha}}^2  \right) }{m_{S_\alpha}^2}h_{2\alpha}^{*}  \right|^2 \,.
\end{align}